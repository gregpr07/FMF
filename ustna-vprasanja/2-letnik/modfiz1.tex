\documentclass[11pt, a4paper]{article}
\usepackage[utf8]{inputenc}
\usepackage[T1]{fontenc}
\usepackage{geometry, microtype, xspace}
\usepackage{parskip}
\usepackage[shortlabels]{enumitem}
\usepackage{amsmath, amssymb, mathtools, bm, esint}
\usepackage{physics, siunitx}
\usepackage{hyperref}
\geometry{margin=3.0cm}
\sisetup{separate-uncertainty=true, exponent-product=\cdot, range-units=single}
\hypersetup{colorlinks=true, linkcolor=blue, urlcolor=cyan}


\begin{document}

\tableofcontents
    
\section{Special Theory of Relativity}
\begin{itemize}

    \item Describe relativistic energy.
    Be sure to mention rest energy, kinetic energy, relativistic energy relation, and the energy-momentum four-vector.

    \item Discuss the center of mass of a system of relativistic particles.
    State and derive the formula for the speed of the center of mass in a laboratory frame, and show how the formula agrees with the classical limit at low speeds.

    \item Discuss the space-time invariance of relativistic quantities.
    Be sure to mention the dot product of four vectors and the specific case of the energy-momentum four vector.

    \item Describe the dynamics of a charged relativistic particle in a uniform electric field and in a uniform magnetic field.
    Describe both the classical and relativistic graphs of particle speed and displacement as functions of time.

\end{itemize}

\section{Semiclassical Quantum Physics}
\begin{itemize}

    \item What is the Compton effect and why can't it be explained with classical physics?
    State and derive the expression for wavelength shift in the Compton effect. 

    \item Discuss X-ray spectra in the context of early quantum physics:
    explain the shape of both the continuous and discrete components of the X-ray spectrum;
    state and derive the minimum wavelength in \textit{bremsstrahlung} radiation;
    what is the expression for the wavelength of the first K-alpha line?

\end{itemize}

\section{Quantum Mechanics in One Dimension}
\begin{itemize}

    \item Discuss the orthnormalization of a quantum particle's energy eigenfunctions.

    \item Discuss linear combinations of eigenfunctions and the concept of expanding a particle's wave function in terms of its energy eigenfunctions?
    Explain the process of computing the coefficients of individual eigenfunctions in the wave function's eigenfunction expansion.

    \item Describe the problem of a quantum particle in an infinite potential well and its relevance to modeling real-life systems.
    State and derive the particle's energy eigenfunctions and eigenvalues;
    why do the eigenfunctions have the form they do?

    \item Describe the problem of a quantum harmonic oscillator and its relevance to modeling real-life systems.
    Summarize the functional form of the oscillator's energy eigenfunctions (no derivation needed) and provide graphs of the first few eigenfunctions.
    What is the expression for a quantum harmonic oscillator's energy eigenvalues? 
    Discuss the correspondence principle and the behavior of a quantum harmonic oscillator in the classical limit.

    \item Describe the problem of an initially-free particle scattering from a finite potential step.
    What is the usual ansatz for the particle's wave function?
    Discuss probability current density and its application to the scattering problem.

\end{itemize}

\section{Atoms and Molecules}
\begin{itemize}

    \item What are electron dipole transitions?
    Discuss the characteristic time in dipole transitions, and give an example calculation for transition times in a infinite potential well.

    \item What is spin-orbit coupling and why/how does it occur?
    What is total angular momentum and how is it relevant in the context of spin-orbit coupling?
    Derive and discuss the splitting of atomic energy levels because of spin-orbit coupling.

    \item Describe vibrational energy transitions in molecules.

    \item Discus rotational energy in molecules:
    describe rotational energy transitions, sketch a typical rotational energy spectrum, and explain the presence of energy gaps in vibrational absorption spectra.

    \item What is the Van der Waals interaction in molecules?
    Describe the role of electric dipoles in the Van der Waals interaction.

\end{itemize}
\end{document}
