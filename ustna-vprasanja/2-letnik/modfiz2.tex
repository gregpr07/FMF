\documentclass[11pt, a4paper]{article}
\usepackage[utf8]{inputenc}
\usepackage[T1]{fontenc}
\usepackage{geometry, microtype, xspace}
\usepackage{parskip}
\usepackage[shortlabels]{enumitem}
\usepackage{amsmath, amssymb, mathtools, bm, esint}
\usepackage{physics, siunitx}
\usepackage{hyperref}
\geometry{margin=3.0cm}
\sisetup{separate-uncertainty=true, exponent-product=\cdot, range-units=single}
\hypersetup{colorlinks=true, linkcolor=blue, urlcolor=cyan}


\begin{document}

\tableofcontents

\section{Solid State Physics}

\subsection{Energy Bands and Electron Conduction}
\begin{itemize}
	\item Describe electron energy levels and their degeneracy in the one-electron atom.
	
	\item What are energy bands in crystals?
    Discuss the Kronig-Penney model and its predictions for energy bands in a one-dimensional crystal.
		
	\item How do energy bands vary in insulators, semiconductors and conductors?

	\item What is the Drude model of electron conduction? 
    Derive the expression of Ohm's law predicted by the Drude model in a constant electric field.
	
	\item Discuss the Drude model of electron conduction in an oscillating electric field.
\end{itemize}

\subsection{Semiconductors}
\begin{itemize}
	\item Describe the number density of electrons in the conduction band and density of holes in the valence band in an intrinsic semiconductor.
    How are the two related?
    What approximations are used when deriving the common expressions for electron and hole number densities?
	
	\item What is a doped semiconductor?
    Discuss the density of electrons in the conduction band and density of holes in the valence band in a doped semiconductor.
    How are the two related?
	
	\item What is a p-n junction?
    Discuss its formation, properties, and applications.
	
\end{itemize}


\section{Nuclear Physics}

\subsection{Nuclear Scattering Experiment}
\begin{itemize}
	\item What was the Rutherford gold foil experiment?
    Discuss the experimental set-up, results and implications for the structure of the atom.
    
    \item Discuss the theory of alpha particle scattering from a nucleus under the Coulomb potential.
	
	\item Discuss the relationship between a particle's momentum and de Broglie wavelength.
	
	\item Explain how a simple multi-stage linear particle accelerator works.
	
\end{itemize}

\subsection{Nuclear Mass}
\begin{itemize}
	\item What is the semi-empirical mass formula?
    Discuss the meaning of each of the five term. 
	
	\item Discuss the nuclear shell model and its predictions for nuclear stability. 
	
	\item Discuss the energy contributions to a nuclear system's Hamiltonian in the shell model.
    Discuss spin-orbit coupling in the shell model.
	
\end{itemize}

\subsection{Nuclear Decay}
\begin{itemize}

    \item Explain the process of alpha decay.
	
	\item Explain the process of beta decay.
	
	\item Explain the process of gamma decay.
	
	\item Discuss in detail the decay of the isotope ${}^{235}\mathrm{U}$.
	
	\item Discuss nuclear fusion and the proton-proton (hydrogen) chain reaction.
	
	\item What is the Lawson criterion for nuclear fusion?
\end{itemize}

\section{Elementary Particles}

\subsection{Elementary Particle Experiments}
\begin{itemize}
	\item Explain how a cyclotron and circular particle accelerator work.
	
	\item Discuss the differences in center-of-mass energy between a fixed-target accelerator and a head-on collision accelerator.
	
	\item How is the momentum of charged particles measured in modern particle detectors?
	
	\item What is Cherenkov radiation?
    Discuss its applications in particle detectors.
\end{itemize}

\subsection{Standard Model}
\begin{itemize}
	\item Which particles are currently thought of as elementary?
    Discuss the basic constituents of the Standard Model for elementary particles.
    What are leptons, baryons, and mesons?
	
	\item Describe the fundamental interactions in terms of exchange particles in the standard model.
	Discuss the mechanism for the electromagnetic interaction in terms of photon exchange.

	\item Which particles carry which interactions and what are the associated coupling constants?
    Discuss the relative ranges and coupling constants strengths for the fundamental interactions.
	
	\item What is the Compton wavelength?
    What is the Compton wavelength for a pion?
	
	\item What is the Yukawa potential? 
	
	\item What is the general form of the matrix element for a process involving interactions carried by massive exchange particles?
	
\end{itemize}

\subsection{Quantum Numbers and Conservation Laws}
\begin{itemize}
	\item What are lepton and baryon numbers?
    Discuss their associated conservation laws and some the implications for elementary particle processes.

	\item What is isospin?
    Discuss isospin in terms of quarks and the pi meson.
    What is the relationship between isospin and charge?
	
	\item Discuss the strangeness quantum number and the motivation for its introduction.
	
	\item Explain the ground states of baryons formed of the $u$ and $d$ quarks with spin $1/2$ in the space of the third component of isospin and hypercharge.
		
	\item Explain the ground states of mesons formed of the $u$ and $d$ quarks with spin $1/2$ in the space of the third component of isospin and hypercharge.
	
	\item Discuss the masses of mesons and barions in the quark model.
	
	\item Discuss experimental evidence for the one-third charge of quarks.
\end{itemize}

\subsection{Weak Interaction}
\begin{itemize}
	\item Provide some examples of decays that occur via the weak interaction and draw the corresponding Feynman diagrams.
	
	\item Discuss the parity, charge conjugation, and helicity operators. 
	
	\item Explain the ${}^{60}\mathrm{Co}$ experiment and its evidence for the weak interaction's violation of parity conservation.
	
	\item Provide and explain an example of the weak interaction violating charge conjugation.
	
	\item Discuss the Cabibbo angle and the CKM matrix.
	
	\item Discuss the phenomenon of neutral meson oscillation.
	
	\item Provide an estimate for the branching ratio for the decay $\tau \to e\nu \overline{\nu}$.
	
	\item What is the Hubble constant?
    Give an overview of the important properties of the expansion of the universe following the big bang.
\end{itemize}

\end{document}











