\documentclass[11pt, a4paper]{article}
\usepackage[utf8]{inputenc}
\usepackage[T1]{fontenc}
\usepackage{geometry, microtype, xspace}
\usepackage{parskip}
\usepackage[shortlabels]{enumitem}
\usepackage{amsmath, amssymb, mathtools, bm, esint}
\usepackage{physics, siunitx}
\usepackage{hyperref}
\geometry{margin=3.0cm}
\sisetup{separate-uncertainty=true, exponent-product=\cdot, range-units=single}
\hypersetup{colorlinks=true, linkcolor=blue, urlcolor=cyan}


\begin{document}

\tableofcontents

\section{Complex Analysis}

\subsection{Holomorphic Functions and the Cauchy Riemann Equations}

\begin{itemize}

	\item What is the definition of a holomorphic function?
	
	\item When is the function $ f(z) = u(x, y) + iv(x, y) $ (where $ u, v \in C^{1}(\mathbb{R}^2)$) holomorphic? Prove your answer. 
	
	\item Provide an example of a function $ f: \mathbb{C} \to \mathbb{C} $ that is differentiable as a real function but is not holomorphic.
	
	\item Discuss and prove: How can a holomorphic function be written as a power series, and where does such a power series converge?
	
	What is the relationship between holomorphic functions $ f:\mathbb{C} \to \mathbb{C} $ and functions $ f $ that may be written as a power series?
	
	How many times is a holomorphic function differentiable? 
	
	 \item How can a holomorphic function be written in integral form? 

\end{itemize}

\subsection{Cauchy Theorem and Integral Formula}

\begin{itemize}

	\item Define the winding number $ \operatorname{Ind}_{\gamma}(z_0) $ of a smooth curve $ \gamma $ about a point $ z_0 $ in the complex plane. 
	
	\item What are some of the winding number's most important properties? State, with proof, what values $ \operatorname{Ind}_{\gamma}(z_0) $ can assume. 
	
	\item Let the curve $ \gamma $ be completely contained in the open disk $ D \subset \mathbb{C} $. What is the value of $ \operatorname{Ind}_{\gamma}(z_0) $ for $ z_0 \in \mathbb{C} \setminus D $? Prove. 

	\item State Cauchy's integral theorem and use Green's formula to prove the theorem for the simplified case when the contour $ \gamma $ does not cross over itself.

\end{itemize}

\subsection{Laurent Series and Isolated Singularities}

\begin{itemize}
	
	\item What is the definition of an isolated singularity of a holomorphic function? 
	
	\item How can an (otherwise) holomorphic function $ f $ be expressed in the neighborhood of an isolated singularity $ z_0 $?
	
	\item Define and discuss a removable singularity, pole, and essential singularity. 
	
	\item Describe, analytically  and with proof, the behavior of a holomorphic function in the neighborhood of an essential singularity. 


\end{itemize}


\subsection{Residue and the Residue Theorem}

\begin{itemize}

	\item State the residue theorem and discuss its applications
	
	\item Calculate $ \int_{-\infty}^{\infty} \frac{e^{cx}}{1+e^{x}}\diff x $ where $ c \in (0, 1) $. (Use a large rectangle as the integration region with two edges on the $ x $ axis and the line $ y = 2\pi $ respectively.)

\end{itemize}

\subsection{Complex Powers and Logarithms}

\begin{itemize}

	\item How are the functions $ z \mapsto \ln z $ and $ z \mapsto z^{\alpha} $ defined in the complex plane?
	
	\item On what regions are the functions $ z \mapsto \ln z $ and $ z \mapsto z^{\alpha} $ holomorphic? Prove.

\end{itemize}

\subsection{Open Mapping Theorem and Maximum Modulus Principle}

\begin{itemize}

	\item What is a meromorphic function and what is an open map?

	\item Discuss bijective meromorphic functions mapping from the expanded complex plane into itself (the M\"{o}bius transform). To what type of curve does such a function map a circle?
	
	Find a bijective holomorphic function of the complex plane into itself that maps the points $ z_1, z_{2}, z_{3} $ into $ 0, 1, \infty $, in that order.
	

	\item State, with proof, the relationship between the difference in the number of zeros and the number of poles poles of a meromorphic function on the unit disk if the function has neither zeros nor poles on the disk's boundary.
		
	Let $ f $ be holomorphic on the unit circle, $ f(0) = -1, f'(0) = 2 $ and define $ a \in (0, 1) $. Evaluate the contour integrals $ \oint_{\abs{z} = a} \frac{f(z)}{z} \diff z $ and  $ \oint_{\abs{z} = a} \frac{f(z)}{z^2} \diff z $. 

	
	\item Discuss and prove the open mapping theorem.
	
	\item Discuss the maximum and minimum modulus principals for holomorphic functions and derive them using the open mapping theorem. What are some implications of the maximum and minimum modulus principle?
	

	\item Discuss the Schwarz lemma for holomorphic mappings between disks that preserve the disk's center, and use the maximum modulus principle to prove the Schwarz lemma.
	
	\item State and  prove how the Schwarz lemma allows us to write holomorphic mappings of the unit disk onto itself.
	
	For the point $ \alpha \in D(0, 1) $, show that the function $ f_{\alpha}(z) = \frac{z - \alpha}{1 - \overline{\alpha}z} $ maps the border $ \partial D(0, 1) $ into itself, then use this result to show that $ f_{\alpha} $ maps $ D \to D $.
		
	Let $ D $ be the unit disk and $ f:D\to D $ be a holomorphic function for which $ f(0) = 0 $. State, with proof, the maximum value of $ \abs{f(z)} $ on $ D $.
		
\end{itemize}

\section{Harmonic Functions}

\subsection{Harmonic Functions in the Plane}

\begin{itemize}

	\item What is the definition of a harmonic function in $ \mathbb{R}^{2} $ and in general?
	
	\item Discuss the relationship between holomorphic functions and harmonic functions in the plane.
	
	\item State the mean value theorem for harmonic functions in the plane. How are the values of a harmonic function in the interior of a disk related to the values on the disk's boundary?
	
	\item What is the maximum modulus principle for harmonic functions in the plane? Use the mean value theorem for harmonic functions to derive the maximum modulus principle.
	
	
	\item Precisely formulate the Dirichlet problem on the unit circle and state its solution in terms of the Poisson formula.

\end{itemize}

\subsection{Harmonic Functions in Space}

\begin{itemize}

	\item Find all harmonic functions in $ \mathbb{R}^{3} $ whose value depends only on the distance $ r $ from the origin.
		
	\item What are Green's identities? State Green's third identity and the mean value property for harmonic functions in space. Use Green's third identity to derive the mean value property.
	
	\item What is the maximum modulus principle for harmonic functions in space? Use the mean value theorem for to derive the maximum modulus principle.
	
	\item Define the Green function for a bounded region with a smooth border in $ \mathbb{R}^{3} $ and discuss how the Green function makes it possible to solve a Dirichlet problem for such a region.
	
\end{itemize}

\section{Fourier Analysis}

\subsection{Convolutions}

\begin{itemize}

	\item State the definition of a convolution of the functions $ f, g: \mathbb{R} \to \mathbb{C} $.
	
	\item State some of the most important properties of convolutions. Show that if $ f, g $ are even functions then $ f*g $ is also even.
	
	\item Discuss how convolutions prove a means to approximate bounded continuous functions with differentiable functions. 
	
	\item What is the Schwartz space of functions? State some of its important properties.

\end{itemize}

\subsection{The Fourier Transform}

\begin{itemize}

	\item How is the Fourier transform defined? 
	
	\item How is the inverse Fourier transform defined? 
	
	\item State some of the Fourier transform's most important properties. What is the Fourier transform of the convolution of two functions? 
	
	\item 	Calculate the Fourier transform of the function $ f(x) = e^{-\frac{x^2}{a}} $ for $ a > 0 $.
	
	Calculate the Fourier transform of the function $ f_{a}(x) = \frac{a}{\pi (x^2 + a^2)} $ where $ a \in \mathbb{R}^{+} $ (use the residue theorem). Derive the identity $ \widehat{f_{a} * f_{b}} = \widehat{f}_{a+b} $ for $ a, b > 0 $
	
\end{itemize}

\section{Differential Equations}

\subsection{Partial Differential Equations}

\begin{itemize}

	\item Derive the d'Alembert formula for the solutions to the one-dimensional wave equation.
	\item Derive how the Fourier transform is used to find solutions to the heat equation $ \pdv{u}{t} = c \pdv[2]{u}{x} $ with the initial condition $ u(x, 0) = f(x) $ where $ f: \mathbb{R} \to \mathbb{R} $ is a continuous function that falls to zero at $ \pm \infty $.

\end{itemize}

\subsection{Zeros of Solutions to Homogeneous Second-Order LDEs}

\begin{itemize}
		
	\item Can the zeros of a nontrivial solution to the equation $ y'' + p(x)y' + q(x)y = 0 $ (where $ p, q \in C(\mathbb{R}) $) have any cluster points? Prove your answer.
	
	\item Discuss the distribution of the zeros of two linearly independent solutions to the equation  $ y'' + p(x)y' + q(x)y = 0 $.
	
	\item How many zeros do the nontrivial solutions $ y $ of the equation $ y'' + (1+x^2)y = 0 $ have? Prove.
	
\end{itemize}

\subsection{Sturm-Liouville Theory}

\begin{itemize}

	\item When is the differential operator $ L(y) = P(x)y'' + Q(x) y' + R(x) y $ formally self-adjoint?
	
	\item For which positive function $ \rho = \rho(x) $ must we multiply the operator $ L(y) = x^{2}y'' + y' $ so that $ L $ is formally self-adjoint?

	\item Precisely formulate the regular Sturm-Liouville problem. 
	
	\item State the Sturm-Liouville theorem and discuss its applications.
	
	\item Discuss the eigenvalues and eigenfunctions and their properties for a regular Strum-Liouville problem. State, with proof, in what sense the eigenfunctions of such a problem are mutually orthogonal.
	
	\item Determine the function $ g $ so that the solutions to the regular Sturm-Liouville problem $ f(x)y'' + g(x)y' + (q(x) + \lambda \rho (x))y $  with $ y(0) = y(1) = 0 $ will be mutually orthogonal with weight $ \rho(x) $ on the interval $ [0, 1] $ for a given $ \lambda $.
	What is the weight for orthogonality for the Sturm-Liouville problem $ (1 + x^2)y'' + xy' + \lambda y = 0 $ with $ y(0) = y(1) = 0 $?

\end{itemize}


\subsection{Series Solutions to Differential Equations}

\begin{itemize}

	\item What is the definition of a proper singular point of the differential equation $ y'' + p(z)y' + q(z)y = 0 $?
		
	\item Discuss the forms of the nontrivial solutions to the equation $ y'' + p(z)y' + q(z)y = 0 $ in the neighborhood of a proper singular point $ z_0 $. In what cases can the second, linearly independent solution to the equation not be expressed with a generalized power series and how is the solution found? What kind of singularity does the second solution have at $ z_0 $ in this case? 
	
	\item When is $ 0 $ a proper singular point of the equation $ y'' + p(z)y' + q(z)y = 0  $?. When is $ \infty $ a proper singular point of the same equation?
	
	\item What is a differential equation's inidcial polynomial and what are teh characteristic exponents?

	\item State the conditions on $ p $ and $ q $ so that the equation has exactly three proper singular points $ z_{1}, z_2, z_3 $. State the conditions on $ p $ and $ q $ in equation (*) so that the differential equation has proper singular points only at $ 0, 1 $ and $ \infty $ and at least one of the characteristic exponents at the singular points $ 0 $ and $ 1 $ equals zero.

\end{itemize}

\subsection{The Legendre and Other Polynomials}

\begin{itemize}

	\item What is the Legendre equation? Discuss the Legendre polynomials and their definition to the Legendre equation.
	
	\item How are the Legendre polynomials defined in terms of the Rodrigues formula? Use the Rodrigues formula to derive the Legendre equation.
	
	\item Discuss, with proof, the orthogonality relation and norms of the Legendre polynomials. 
	
	\item How are the Laguerre polynomials defined?
	
	\item Derive the second order homogeneous LDE solved by the Laguerre polynomials $L_{n}(z) = \frac{e^{z}}{n!} \dv[n]{}{z} (z^{n}e^{-z})$.
	
	\item What are the Hermite polynomials? What equation to they solve? 
    Derive the homogeneous 2nd order linear differential equation solved by the Hermite polynomials $H_{n}$. 
	
	\item How are the Chebyshev polynomials defined?
    Derive the homogeneous second-order LDE solved by the Chebyshev polynomials $ T_{n}(z) $.

\end{itemize}

\subsection{Bessel Equation and Bessel Functions}

\begin{itemize}

	\item What is the Bessel equation? Discuss the solutions.
	
	\item Use the Bessel equation to derive the series for the Bessel functions.
	
	\item What is the generating function for the Bessel functions $ J_{n} $ for $ n \in \mathbb{N} $? State and derive how the generating function is expressed in terms of the Bessel functions.
	
	\item Use the generating function to derive the integral formula for the Bessel functions $ J_{n} $.
	
\end{itemize}

\end{document}






