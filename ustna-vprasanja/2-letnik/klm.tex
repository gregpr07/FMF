\documentclass[11pt, a4paper]{article}
\usepackage[utf8]{inputenc}
\usepackage[T1]{fontenc}
\usepackage{geometry, microtype, xspace}
\usepackage{parskip}
\usepackage[shortlabels]{enumitem}
\usepackage{amsmath, amssymb, mathtools, bm, esint}
\usepackage{physics, siunitx}
\usepackage{hyperref}
\geometry{margin=3.0cm}
\sisetup{separate-uncertainty=true, exponent-product=\cdot, range-units=single}
\hypersetup{colorlinks=true, linkcolor=blue, urlcolor=cyan}


\begin{document}

\tableofcontents

\section{Newtonian Mechanics}

\begin{itemize}

    \item Summarize the Newtonian dynamics of a single particle.

    \item Summarize the Newtonian dynamics for a system of $ N $ particles with masses $ m_1, m_2, \dots, m_N$ and position vectors $ \bm{r}_1, \bm{r}_2, \dots, \bm{r}_{N} $.

    \item Describe the necessary modifications for Newton's laws to hold in non-inertial systems.

\end{itemize}

\section{Lagrangian Mechanics}

\begin{itemize}

    \item Explain the role of constraints and generalized coordinates in Lagrangian mechanics.

    \item Explain virtual displacements and d'Alembert's principle.

    \item Derive the Euler-Lagrange equations from d'Alembert's principle.

    \item What is the principle of least action? Use the least action principle to derive the Euler-Lagrange equations.

    \item What is the definition of a conserved quantity in Lagrangian mechanics? What is a continuous symmetry of the Lagrangian function? Give examples of each. State and prove (the simple version of) Noether's theorem, and provide examples of its application.

    \item Explain the basics of the Lagrangian formalism for a continuous body. Include a derivation of the Lagrange equations for a one-dimensional continuous body.

\end{itemize}

\section{Central Forces}

\begin{itemize}

    \item What is the one-body central force problem? Discuss the problem's solution in both Newtonian and Lagrangian mechanics, making sure to explain the role of conserved quantities. What is the orbit equation? Discuss.

    \item Discuss two-body central force problem. Explain how the problem is solved in both Newtonian and Lagrangian mechanics. Be sure to explain the role of conserved quantities in reducing the two-body problem to an equivalent one-body problem.

    \item What is the Kepler problem? Discuss the relationship between the Kepler problem and the orbit equation. Solve the orbit equation, and discuss its implications for the orbits of planets. State and derive Kepler's laws of planetary motion.

    \item What is the Laplace-Runge-Lenz vector and how is it derived? Discuss the role of the LRL vector in solving the orbit equation.

\end{itemize}

\section{Motion of Rigid Bodies}

\begin{itemize}

    \item Explain how the Euler angles are used to encode the position and kinematics of a rotating rigid body.

    \item Provide a thorough treatment of the kinematics of a rigid body with on fixed point. Be sure to mention angular velocity, the inertia tensor, angular momentum, and the Euler equations.

    \item Discuss and analytically solve the equations of motion for a symmetric free top, then discuss the stability an asymmetric free top using a perturbation approach.

    \item Analyze the basic dynamics of a heavy symmetric top with Lagrangian mechanics. Be sure to discuss the role of conserved quantities in simplifying the solution process. Discuss the conditions for uniform precession and a sleeping top.

\end{itemize}

\section{Small Oscillations}

\begin{itemize}

    \item Discuss the eigenfrequencies and normal modes of small oscillations about an equilibrium position in a system with $N$ degrees of freedom.

    \item What are normal coordinates and how are they relevant to small oscillations?

\end{itemize}

\section{Hamiltonian Mechanics}

\begin{itemize}

    \item Give an overview of the Hamiltonian formulation of mechanics. Be sure to discuss the relationship between the Lagrangian and Hamiltonian formalism via the Legendre transform. Discuss conserved quantities and the least action principle in the Language of Hamiltonian mechanics.

    \item Solve the equations of motion for a charged particle in an electromagnetic field using both Lagrangian and Hamiltonian formulation of mechanics.

    \item What are Poisson properties? State their basic properties and discuss, with examples, their role in Hamiltonian mechanics.

\end{itemize}

\end{document}





