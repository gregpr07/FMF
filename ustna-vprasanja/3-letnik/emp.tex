\documentclass[11pt, a4paper]{article}
\usepackage[utf8]{inputenc}
\usepackage[T1]{fontenc}
\usepackage{geometry, microtype, xspace}
\usepackage{parskip}
\usepackage[shortlabels]{enumitem}
\usepackage{amsmath, amssymb, mathtools, bm, esint}
\usepackage{physics, siunitx}
\usepackage{hyperref}
\geometry{margin=3.0cm}
\sisetup{separate-uncertainty=true, exponent-product=\cdot, range-units=single}
\hypersetup{colorlinks=true, linkcolor=blue, urlcolor=cyan}


\begin{document}

\section{Electrostatics}

\begin{itemize}

    \item Define electric field lines and electric circulation, and discuss the important properties of electric field lines. State and derive the first Maxwell equation in the regime of electrostatics and discuss its relationship to both electric field lines and electric potential.

    \item State and derive Gauss's law (the second Maxwell equation) in both integral and differential form, then use Gauss's law to derive the Poisson equation for electrostatic potential.

    \item Provide examples of a few common charge distributions.

    \item State the solution to the Poisson equation for a point particle and for a generalized charge distribution. State and derive the Green's function for the Poisson equation.

    \item State and derive the electrostatic energy of a charge distribution in an external electric field.

    \item State and derive the total electrostatic energy in an electric field. Give the expression for total electric field energy in terms of both the scalar potential $ \phi $ and the electric field $ \bm{E} $.

    \item State and derive the force on a charge distribution in an external electric field in terms of the electrostatic stress tensor. Use the result to calculate the electric force between two point charges of a.) equal and b.) opposite charge.

    \item State and derive the multipole expansion of the electric potential up to the first order term, and use the result to define electric dipole moment, and the electric potential and electric field of an electric dipole.

    \item State and derive the multipole expansion of electrostatic energy up to the dipole term for a charge distribution in an external electric field. Use the result to derive the force and torque on an electric dipole in an external electric field.

\end{itemize}

\section{Magnetostatics}

\begin{itemize}

    \item Define electric current density. State the electric current densities for current along a current-carrying wire and for a continuous charge distribution moving through space.

    \item State and derive the Biot-Savart law for the magnetic field of a continuous charge distribution. Use the result to derive the magnetic field of a straight conducting wire.

    \item State the relationship between magnetic flux density (i.e. the $ \bm{B} $ field) and electric current density. Define and discuss magnetic field lines.

    \item State the expression for the magnetic vector potential and discuss the mathematical foundation for its definition. How is magnetic flux written in terms of vector potential?

    \item State and derive the magnetic analog of the Poisson equation relating magnetic potential and current density. State the general solution, and discuss its relationship to the Biot-Savart law.

    \item State the magnetic vector potential inside and outside a long, straight inductor. Give an overview of the derivation process, and explain how it is possible to simplify the magnetic field outside the inductor using gauge transformations.

    \item State and derive the expression for the magnetic energy of a charge distribution in an external magnetic field.

    \item Derive the expression for the total magnetic field energy associated with a magnetic field. Give the result both in terms both in terms of magnetic field and magnetic vector potential. Discuss the differences between finding the total magnetic field energy of a magnetic field and finding the total electric field energy of an electric field.

    \item State and derive the magnetic force on current distribution in an external magnetic field in terms of both magnetic field and the magnetostatic stress tensor. Use the result to calculate the magnetic force between two straight, parallel conducting wires carrying an electric current in (a) equal directions and b.) opposite directions.

    \item State and derive the multipole expansion of the magnetic vector potential to the dipole term. Use the result to define magnetic dipole moment, and write the multipole expansion in terms of magnetic moment.

    \item State and derive the magnetic field of a magnetic dipole. Explain the concept of Ampere equivalence.

    \item State and derive the multipole expansion of magnetic energy up to the dipole term. Use the result to derive the force and torque on a magnetic dipole in an external electric field.

\end{itemize}

\section{Quasistatic Electromagnetic Fields}

\begin{itemize}

    \item State the Maxwell equations in the regime of quasistatic electromagnetic fields, and derive the equations that are different from their static analogs. Show that in the quasistatic regime, the Maxwell equations correspond to closed current loops.

    \item Define canonical momentum in the context of electromagnetism, and explain the motivation for its definition.

    \item State and derive the complete relationship between electric field strength and the electric and magnetic potentials. Use the result to calculate the curl of the electric field and interpret the result.

    \item State Ohm's law in terms of electric field and current density. Discuss the basic properties of conductors, and use Ohm's law to explain the behavior of the electric field within a conductor and the electric potential on the conductor's surface.

    \item Discuss the concept of a conductor's relaxation time. What does this time represent, how is it related to electrical conductivity, and what is a typical order of magnitude?

    \item Derive Ohm's law from the Drude model of electrical conduction. Discuss the Drude model's prediction for a material's electrical conductivity. 

    \item What is capacitance? State and derive the expression for the capacitance of an arbitrary configuration of $ N $ conductors. How is the generalized form of capacitance related to a capacitive system's total electric field energy?

    \item What is inductance? State and derive the expression for the inductance of an arbitrary configuration of $ N $ conductors. How is the generalized form of inductance related to a inductive system's total magnetic field energy and the magnetic flux through the inductive system?

    \item What is the skin effect? State the equations needed to analyze the skin effect, give their solutions in the case of a cylindrical conductor, and discuss the results. Discuss the limit cases of complex impedance for small and large frequencies.

\end{itemize}

\section{Maxwell Equations}

\begin{itemize}

    \item State and derive the Maxwell equations in free space. Discuss the mathematical foundation of the Maxwell equations with reference to the Helmholtz decomposition theorem.

    \item Define and state the physical meaning of the Poynting vector and electromagnetic energy density.
    State, derive, and interpret the Poynting theorem for conservation of electromagnetic energy in free space.

    \item State and interpret the Cauchy continuity equation encoding conservation of electromagnetic momentum in both differential form, and sketch the equation's derivation.

\end{itemize}

\section{Electromagnetic Fields in Matter}

\begin{itemize}

    \item Define and discuss free and bound charge, polarization and electric susceptibility. State the definition of the $ \bm{D} $ field, discuss some of its important properties, and explain its role in the study of electric fields in matter.

    \item Define bound current density, magnetization and magnetic susceptibility. State the definition of the $ \bm{H} $ field, discuss some of its important properties, and explain its role in the study of magnetic fields in matter.

    \item State Maxwell's equations in matter. State and discuss the constitutive relations for the electric and magnetic fields in the linear regime.

    \item How does the Poynting theorem for conservation of electromagnetic energy generalize to matter? How do electric and magnetic field energies change in matter compared to free space?

    \item Derive an expression for the electrostatic force on a nonhomogeneous material, and discuss the limiting cases of a dielectric material in the quasistatic regime.

    \item State and derive the boundary conditions for the Maxwell equations along the boundary between two materials with different electromagnetic properties.

\end{itemize}

\section{Frequency Dependence of the Dielectric Constant}

\begin{itemize}

    \item State and derive the relationship between electric field, polarization, and the dielectric function in the frequency domain. Discuss the physical significance of the dielectric function's real and imaginary components in the frequency domain.

    \item Discuss the Debye, Lorentz and plasma models of dielectric relaxation and derive and sketch the associated expressions for the real and imaginary components of the dielectric function as a function of electric field frequency. Provide an example of a real-world material's dielectric function.

    \item State and discuss the Kramers-Kronig relations for the dielectric function.

    \item State and derive the expression for the dissipation of electric field energy in matter in terms of the dielectric function in the frequency domain.

\end{itemize}

\section{Hamiltonian Formalism for the Electromagnetic Field}

\begin{itemize}

    \item State and derive the Lagrange function for a charged particle in an electromagnetic field.

    \item State and derive the Hamiltonian function for a charged particle in an electromagnetic field.

    \item State and derive the complete Lagrangian density function of a continuous charge distribution in an external electromagnetic field. Discuss the associated action, Euler-Lagrange equations, and the resulting Riemann-Lorenz equations.

\end{itemize}

\section{Introduction to Special Relativity}

\begin{itemize}

    \item Briefly discuss the Loretnz transformation between two frames of reference moving relative to each other along a mutual $ x $ axis. State and sketch the derivation of the Lorentz transformation of the electric and magnetic field between two frames of reference. Demonstrate the implications of the above transformations on how the quantities $ \bm{E} \cdot \bm{B} $ and $ E^{2} - c^{2}B^{2} $ transform between frames of reference, and provide a physical interpretation of the results.

    \item Briefly discuss Minkowski space, the concept of covariant and contravariant vectors, and the generalization of the scalar product to Minkowski space. Discuss conservation of charge in Minkowski space and derive the associated expression for the current density four vector. Show that the magnitude of the current density four vector is invariant under Lorentz transformations.

    \item Derive the expression for the electromagnetic potential four vector in terms of the Riemann-Lorenz equations and the current density four vector. Show that the magnitude of the electromagnetic potential four vector is invariant under Lorentz transformations.

    \item State the electromagnetic tensor in both component and matrix form and sketch the motivation for its definition.

\end{itemize}

\end{document}
