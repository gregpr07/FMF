\documentclass[11pt, a4paper]{article}
\usepackage[utf8]{inputenc}
\usepackage[T1]{fontenc}
\usepackage{geometry, microtype, xspace}
\usepackage{parskip}
\usepackage[shortlabels]{enumitem}
\usepackage{amsmath, amssymb, mathtools, bm, esint}
\usepackage{physics, siunitx}
\usepackage{hyperref}
\geometry{margin=3.0cm}
\sisetup{separate-uncertainty=true, exponent-product=\cdot, range-units=single}
\hypersetup{colorlinks=true, linkcolor=blue, urlcolor=cyan}


\begin{document}

\section{Part 1}
\begin{itemize}

    \item Geometric optics: describe and derive the ray equation and common transfer matrices.

    \item Describe the wave equations and its solutions in homogeneous, isotropic and nonconducting material.

    \item Properties of electromagnetic waves in conducting material.

    \item Describe polarization of electromagnetic waves and the Jones calculus.

    \item Describe the passage of electromagnetic waves across an interface (reflection and refraction) for TE-polarized waves.

    \item Describe the passage of electromagnetic waves across an interface (reflection and refraction) for TM-polarized waves.

    \item Describe total internal reflection of electromagnetic waves on the border between two materials.

    \item Describe reflection and refraction on metallic materials.

    \item Explain Fraunhoffer diffraction.

    \item Explain Fresnel diffraction.

    \item Describe the scattering of light in non-homogeneous materials.

\end{itemize}

\section{Part 2}

\begin{itemize}

    \item Interference: describe Young's experiment and related experiments.

    \item Interference: describe the Michelson interferometer and related interferometers/optical apparatuses.

    \item Describe the Fabry-Perot interferometer and the properties of interference on thin films.

    \item Describe temporal coherence.

    \item Describe spatial coherence.
    
    \item The Lorentz model for frequency dependence of the refractive index.

    \item Describe optically active materials and the magneto-optic (Faraday) effect.

    \item Properties of electromagnetic waves in anisotropic materials: the index ellipsoid and wave vector surface.

    \item Describe the refraction of electromagnetic waves on the boundary between an isotropic and optically uniaxial material.

    \item Describe some common optical elements built using anisotropic materials.

    \item Describe optical amplification, the occupation equations, and the basic working principles of a laser.

\end{itemize}

\end{document}

