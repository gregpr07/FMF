\documentclass[11pt, a4paper]{article}
\usepackage[utf8]{inputenc}
\usepackage[T1]{fontenc}
\usepackage{geometry, microtype, xspace}
\usepackage{parskip}
\usepackage[shortlabels]{enumitem}
\usepackage{amsmath, amssymb, mathtools, bm, esint}
\usepackage{physics, siunitx}
\usepackage{hyperref}
\geometry{margin=3.0cm}
\sisetup{separate-uncertainty=true, exponent-product=\cdot, range-units=single}
\hypersetup{colorlinks=true, linkcolor=blue, urlcolor=cyan}


\begin{document}

\tableofcontents

\section{Fundamentals of Quantum Mechanics}

\begin{itemize}

    \item What is the Schrödinger equation?
    Explain some reasons why the Schrödinger equation (and not some other differential equation) forms the basis of quantum mechanics.

    \item What is a wave function and what is its role in quantum mechanics? How is the wavefunction related to the probability of detecting a particle in a region of space? 

    \item What are stationary states? Include a  physical interpretation. State and derive the stationary Schrödinger equation and discuss its relationship to stationary states. Discuss the relationship between the stationary Schrödinger equation and time evolution.

    \item Discuss some of the wave function's most important properties.

    \item Explain the role of operators in quantum mechanics and state some of their important properties. Discuss the momentum operator.

    \item Define and explain the concept of an expectation value of an operator in the context of quantum mechanics. How do we find the time derivative of an expectation value? 

    \item How is the momentum operator defined in one and three dimensions? How is the momentum operator related to probability current? Explain the motivation behind the definition of the momentum operator.

    \item State and derive the Ehernfest theorem and explain the theorem's physical significance.

    \item State and derive the virial theorem in quantum mechanics.

\end{itemize}

\section{The Dirac Formalism}

\begin{itemize}

    \item State the postulates of the Copenhagen interpretation of quantum mechanics.

    \item Explain Dirac braket notation and its relationship to Hilbert spaces. State the definition and notation of the inner product in the Hilbert space of wave functions, along with some of the inner product's most important properties.

    \item State and discuss how a wavefunction and an operator are expanded in a given orthonormal basis. How is the identity operator written in terms of an orthonormal basis? How is an operator equation written in an orthonormal basis? Provide a matrix interpretation.

    \item What is the definition of a Hermitian and anti-Hermitian operator? Discuss the important properties of Hermitian operators.

    \item What is the definition of an adjoint operator? Discuss some the important properties of adjoint operators. How is an adjoint operator written in an orthonormal basis?

    \item How are unitary and anti-unitary operators defined? State some of their important properties, and discuss the role of unitary operators in quantum mechanics.

    \item What are the momentum eigenfunctions? How are they normalized? State and derive how differentiation in $ x $ space is related to multiplication in $ p $ space, and vice versa.

    \item State, derive, and interpret the meaning of the quantities $ \braket{p}{\psi} $ and $ \braket{x}{\psi} $.
    How is a state $ \ket{\psi} $ expanded in the momentum and position eigenstates in Dirac notation?

\end{itemize}

\section{Important Quantum Systems}

\begin{itemize}

    \item Discuss the quantum harmonic oscillator. How are the ladder operators $ a $ and $ a^{\dagger} $ defined, and how are the position, momentum and Hamiltonian operators written in terms of the ladder operators? Discuss the matrix forms of the ladder operators, position, momentum, and Hamiltonian.

    \item State the harmonic oscillator's eigenvalues and eigenstates, and explain the derivation process using the algebraic ladder operator method. Give special attention to the ground state eigenfunction and eigenvalue.

\end{itemize}

\section{Symmetries}

\begin{itemize}

    \item How is the quantum mechanical translation operator defined in one and three dimensions? What is the operator's generator? Which problems have translational symmetry?

    \item State and derive the quantum-mechanical rotation operator. What is the operator's generator? Which problems have rotational symmetry?

    \item Discuss the quantum-mechanical parity operator and give a physical interpretation of parity transformation. State and derive some of the parity operator's important quantities. Discuss the relationship of the parity operator to problems with even potentials.

    \item How is the time-reversal operator defined? Discuss the time reversal in the context of problems with time-independent potentials.

    \item Discuss a wavefunction's invariance under phase change in the context of gauge transformations. How do a wavefunction and basis change under a global phase or potential energy shift?

\end{itemize}

\section{Angular Momentum}

\begin{itemize}

    \item Discuss the definition and basic properties of angular momentum in quantum mechanics. Be sure to include the important angular momentum commutation relations. How is the squared angular momentum operator defined?

    \item Define the ladder operators for angular momentum, and discuss some of their important properties.

    \item State and derive the angular momentum eigenfunctions, eigenvalues and eigenbasis. Discuss the degeneracy of the $ L^{2} $ eigenvalue spectrum.

\end{itemize}

\section{Central Potential}

\begin{itemize}

    \item State the basis problem of a particle in a central potential. Explain the general solution procedure, and state and derive the radial eigenvalue equation.

    \item State and derive the solution to the radial eigenvalue equation in the limit $ r \to 0 $. You may restrict your analysis to potentials obeying the limit $ \lim_{r \to 0} r^{2}V(r) = 0. $

    \item State, derive and interpret the solution to the radial eigenvalue equation in the limit $ r \to \infty $. You may restrict your analysis to potentials vanishing at infinity, but be sure to formally discuss the validity of the results.

    \item Discuss the general solution procedure for the quantum-mechanical problem of a bound electron in a Coulomb potential. State and derive the relevant radial eigenvalue equation.

\end{itemize}

\section{Charged Particle in EM Field}

\begin{itemize}

    \item State and derive the Schrödinger equation for a charged particle in an electromagnetic field. Be sure to consider the simplification resulting from using the Coulomb gauge.

    \item Define and interpret the magnetic dipole moment operator and the Bohr magneton. Discuss the coupling of magnetic moment to a homogeneous external magnetic field, defined the Zeeman coupling term, and show that in a homogeneous field the quadratic coupling term is usually negligible in comparison to the Zeeman term. Explain and discuss the normal Zeeman effect.

    \item Explain Landau quantization and Landau levels and give a physical interpretation. Define the Landau gauge potential. Restrict your analysis to the $ xy $ plane.

    \item Discuss gauge transformations in the context of a particle in an electromagnetic field.

    \item Qualitatively describe the Aharonov-Bohm effect, and then derive the formalism needed to explain it quantitatively. Be sure to discuss the role of gauge transformations.

\end{itemize}

\section{Spin}

\begin{itemize}

    \item Define spin, explain its relationship to angular momentum, and state some of the important spin properties.

    \item Discuss the basis properties of a particle with spin $ s = 1/2 $, explain why such particles are physically important, and define the up and down arrow notation $ \ket{\uparrow} $ and $ \ket{\downarrow} $.

    \item Define the Pauli spin matrices and discuss some of their important properties. State how the common spin operators are written in terms of the Pauli spin matrices.

    \item Discuss spinors in the context of particles with spin $ s = 1/2 $, and discuss how spinors are transformed by rotation and time reversal. Describe the process of changing the axis of quantization from the conventional $ z $ axis to an arbitrary direction in space.

    \item What is the definition of spin magnetic moment? State the coupling term for spin-orbit coupling, and explain how the result is derived.

    \item Explain the outcome of the Stern-Gerlach experiment, including a formal derivation. Explain the implications of the experiment's results on quantization of magnetic moment and spin.

\end{itemize}

\section{Addition of Angular Momentum}

\begin{itemize}

    \item Explain the formalism for addition of angular momentum in quantum mechanics for a system of two particles with spin $ s = 1/2 $.

    \item State and derive the singlet and triplet states for a system of two particles with spin $ s = 1/2 $.

    \item Explain the Heisenberg coupling interaction between two particles with spin $ s = 1/2 $. Be sure to discuss the relevant eigenvalue equation and energy eigenvalues.

    \item What are the Clebsch-Gordan coefficients? From where do they arise? Explain what they are used for.

\end{itemize}

\section{Perturbation Theory}

\begin{itemize}

    \item Discuss the Rayleigh-Schrödinger method for first-order perturbative analysis of a quantum system with a non-degenerate spectrum. Be sure to state and derive the relevant eigenvalue and eigenfunction formulas.

    \item Explain the process for perturbatively finding a degenerate system's energy eigenvalues. Demonstrate the process for a doubly degenerate energy level, and use the result to generalize the procedure to arbitrary degeneracy.

    \item Explain the process for perturbatively finding the eigenfunctions of a system with a time-dependent Hamiltonian.

    \item Discuss the probability for transition between states under the influence of a step-like potential. State, derive and interpret both the general first-order perturbative formula and Fermi's golden rule, and explain in which situations the latter applies.

    \item Provide an overview of the \emph{WKB} method in the context of quantum mechanics. Define the \emph{WKB} approximation, and explain how it connects quantum and classical mechanics.

    \item Explain the variational method and what it is used for. State and derive the relevant equations.

\end{itemize}

\section{Scattering}

\begin{itemize}

    \item Discuss the analysis of the one-dimensional scattering problem. Be sure to discuss the general ansatz for stationary scattering states, the scattering matrix, and the transfer matrix.

    \item Discuss the one-dimensional scattering problem for a systems invariant under time reversal and parity transformation. State and derive how the scattering matrix simplifies.

    \item Discuss scattering states and the various methods for normalizing plane waves.

    \item Give an overview of the generalization of the scattering problem to three dimensions. Discuss the important differences between the one and three-dimensional cases, and discuss the expansion of scattering states in terms of spherical waves.

    \item What are the scattering cross section and differential cross section? Discuss their definition and physical interpretation. State and derive the optical theorem.

\end{itemize}

\end{document}

