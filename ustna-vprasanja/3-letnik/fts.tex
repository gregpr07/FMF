\documentclass[11pt, a4paper]{article}
\usepackage[utf8]{inputenc}
\usepackage[T1]{fontenc}
\usepackage{geometry, microtype, xspace}
\usepackage{parskip}
\usepackage[shortlabels]{enumitem}
\usepackage{amsmath, amssymb, mathtools, bm, esint}
\usepackage{physics, siunitx}
\usepackage{hyperref}
\geometry{margin=3.0cm}
\sisetup{separate-uncertainty=true, exponent-product=\cdot, range-units=single}
\hypersetup{colorlinks=true, linkcolor=blue, urlcolor=cyan}


\begin{document}

\tableofcontents

\section{Crystal Structure}

\subsection{Bravais Lattice}
\begin{itemize}

    \item Definition of a Bravais lattice, primitive unit cell, and primitive vectors.

    \item Symmetry operations on a lattice, e.g. translations and point symmetry operations.

    \item Examples and descriptions of the more common Bravais lattices.

    \item Classifying Bravais lattices.

\end{itemize}

\subsection{The Reciprocal Lattice}
\begin{itemize}

    \item Definition and formulation of the reciprocal lattice.

    \item Lattice planes and the Miller indices.

    \item The Bragg formulation of X-ray scattering from a crystal lattice.

    \item The von Laue formulation of X-ray scattering from a crystal lattice.

    \item The geometric structure factor and its significance in X-ray scattering.

\end{itemize}

\section{Electrons in Crystals}

\subsection{Free Electron Model}

\begin{itemize}

    \item The free electron model.

    \item The temperature dependence of specific heat in the free electron model.

    \item The temperature dependence of chemical potential in the free electron model.

\end{itemize}

\subsection{Periodic Potential}
\begin{itemize}

    \item Electrons in a periodic potential and Bloch's theorem.

    \item The nearly-free electron model; 
    electron states in a periodic potential in the nearly-free electron model.

    \item Density of electron energy levels.

    \item The tight-binding model.

    \item The semiclassical description of electron motion in a periodic potential; effective electron mass and effective hole mass.

    \item Bloch oscillations; semiclassical description of electron motion in the presence of an external magnetic field.

\end{itemize}


\section{Semiconductor Devices}

\subsection{Semiconductors}
\begin{itemize}

    \item Definition and properties of homogeneous semiconductors;
    the difference between conductors, semiconductors and insulators.

    \item Number density of electrons in the conducting band and density of holes in the valence band for a homogeneous, intrinsic semiconductor.

    \item The temperature dependence of the chemical potential in an intrinsic semiconductor.

\end{itemize}

\textbf{Doped Semiconductors}
\begin{itemize}

    \item Dopant energy levels and their effect on the density of charge carriers in the conducting and valence bands.

    \item Occupation of donor and acceptor energy levels in doped semiconductors;
    equilibrium charge carrier densities in doped semiconductors.

\end{itemize}

\subsection{Semiconductor Junctions}
\begin{itemize}

    \item Definition and properties of non-homogeneous semiconductors;
    the formation of the depletion region.

    \item Dependence of the depletion region potential on donor and acceptor densities;
    dependence of potential difference on the carrier concentrations $N_{\text{d}}$ and $P_{\text{a}}$;
    the shape of charge carrier energy bands near a p-n junction.

    \item Derivation of the potential difference in a p-n junction in the limit of fully-ionized dopant energy levels

    \item Properties of a p-n junction exposed to external bias voltage;
    the relationship between bias voltage and width of depletion region

    \item A p-n junction's $I(V)$ (current-voltage) characteristic

\end{itemize}

\section{Lattice Oscillations}
\begin{itemize}

    \item One-dimensional model of lattice oscillations with one atom per unit cell;
    the concept of a frequency vs. wave vector dispersion relation;
    the connection between lattice oscillations and the wave equation.

    \item One-dimensional model of lattice oscillations with two atoms per unit cell;
    the optical and acoustic branches of the dispersion relation.

    \item Lattice oscillations at points of high reciprocal lattice symmetry (e.g. $k=0$ and $k=\pi$);
    the optical and acoustic branches of the dispersion relation in the one-dimensional lattice oscillations with two-atoms per unit cell.

    \item The dispersion relation and description of eigenmodes in three dimensions in crystal structures with both one and multiple atoms per unit cell.

    \item Quantization of lattice oscillations;
    formulation of the Hamiltonian for one-dimensional lattice oscillations in terms of the boson annihilation and creation operators.

    \item The contribution of phonons to specific heat and its low-temperature and high-temperature limits.

    \item The Debye approximation for the contribution of phonons to specific heat.

    \item Consequences of anharmonic corrections in the expansion of a lattice's elastic energy, including the thermal expansion of crystals.

    \item The kinetic theory of heat condition in a crystal lattice.

    \item Umklapp processes and the temperature dependence of heat conduction in a crystal lattice.

\end{itemize}
    
\end{document}

