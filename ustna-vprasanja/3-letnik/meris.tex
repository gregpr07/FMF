\documentclass[11pt, a4paper]{article}
\usepackage[utf8]{inputenc}
\usepackage[T1]{fontenc}
\usepackage{geometry, microtype, xspace}
\usepackage{parskip}
\usepackage[shortlabels]{enumitem}
\usepackage{amsmath, amssymb, mathtools, bm, esint}
\usepackage{physics, siunitx}
\usepackage{hyperref}
\geometry{margin=3.0cm}
\sisetup{separate-uncertainty=true, exponent-product=\cdot, range-units=single}
\hypersetup{colorlinks=true, linkcolor=blue, urlcolor=cyan}


\begin{document}

\tableofcontents

\section{Particles in Matter}
\begin{itemize}

    \item Describe the principal processes in the interaction of heavy charged particles with matter.
    Derive the Bohr equation for a heavy charged particle's ionizing energy losses in matter.
    Describe the quantum-mechanical corrections which generalize the Bohr equation to the Bethe-Bloch equation

    \item What is a Bragg curve? 
    Discuss the quantities affecting a charged particle's range in matter.

    \item Describe the principal processes in the interaction of photons with matter.

\end{itemize}

\section{Gas-Based Detectors}

\begin{itemize}

    \item Describe the working principles of cylindrical gas-based detector of ionizing particles.
    Discuss the detector's various working regimes.

    \item What is the multiplication factor in proportional gas detectors (counters)?
    How is the multiplication factor in proportional counters related to the multiplication factor in a Geiger-Müller tube

    \item Describe the mechanisms for ionization and recombination in gas.

    \item Describe the relationship between the mean number of ion pairs created in a detector and the detector's energy resolution. 
    What is a detector's Fano factor?

    \item Describe the diffusion process of electrons and positive ions in gas in an external electric field.
    Describe the spatial distribution of charge avalanches in a gas detector.

    \item Derive the time-dependent voltage signal on the anode of a proportional cylindrical ionization detector in response to an incident ionizing particle.

    \item Describe the working principles of an analog differentiation circuit.
    Discuss how a differentiation circuit is used to modify the signal from a cylindrical proportional counter.

    \item Describe the working principles of a multi-wire proportional counter.

    \item Describe the working principles of a drift chamber.

\end{itemize}

\section{Semiconducting Detectors}

\begin{itemize}

    \item Describe the basic properties of the semiconductors used in particle detectors. 
    Explain the formation of the depletion region in a p-n junction.
    Describe the spatial distribution of charge, electric potential and electric field in a p-n junction's depletion region.

    \item What is leakage current?
    Describe how metal electrodes are placed on semiconductor detectors.

    \item Derive the time-dependent signal on the electrons of a semiconductor detector in response to an incident particle freeing an electron-hole pair

    \item What are diffused-junction diodes (\textit{difundirane diode})?
    Describe their advantages and disadvantages for particle detection.

    What is a Schottky barrier, and how is it used in semiconductor detectors?
    Discuss advantages and disadvantages for particle detection.

    \item What is a p-i-n detector (e.g. an Si-Li detector)? 
    Describe some of a p-i-n detector's important properties.

    \item What is a Ge-Li detector?
    Outline its construction and important properties.

    \item Describe position-sensitive semiconductor detectors and their important properties.

\end{itemize}

\section{Scintillating Detectors}
\begin{itemize}

    \item What are scintillators?
    Describe their general properties.

    \item Describe the scintillation mechanism in organic scintillators. 
    Describe organic scintillators' important properties, and give examples of typical organic scintillators.

    \item Describe the scintillation mechanism in inorganic scintillators.
    Describe inorganic scintillators' important properties, and give examples of typical inorganic scintillators.

    \item Discuss response linearity in scintillating detectors.

    \item Describe the working principles of a photomultiplier tube.
    Describe the important properties of each of PMT's components.

    \item Describe some of the techniques and challenges involved in neutron detection.

\end{itemize}

\section{Particle Identification}
\begin{itemize}

    \item Discuss the working principles of time-of-flight detectors. 
    Discuss how time-of-flight detectors can be used to classify kaons and pions.

    \item Discuss particle identification based on multiple measurements of ionizing energy losses.

    \item What is Cherenkov radiation?
    What are typical radiating materials used for Cherenkov detectors?

    \item Discuss the working principles of a threshold Cherenkov detector.

    \item Discuss the working principles of a ring-imaging Cherenkov detector.

    \item What is transition radiation? How can transition radiation be used for particle identification?

    \item Summarize which particle identification methods are useful in which particle momentum regimes.
    State in which regime each method is limited and explain why.

\end{itemize}

\section{Radiation Safety}
\begin{itemize}

    \item Discuss the time dynamics of radioactive decay.
    Explain the meaning of activity, decay time, and half-life?

    \item Describe the categorization of the biological effects of radiation exposure into deterministic and stochastic effects.
    Discuss representative consequences in each category.

    \item Explain the differences between absorbed, equivalent and effective dose.

    \item Outline radiation safety protocols.
    Give typical values of maximum doses allowed for different occupations.

\end{itemize}

\end{document}
